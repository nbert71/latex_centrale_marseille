% Template LaTeX écrit par Nicolas BERT sur la base des templates de Centrale Nantes, Supélec et Lyon.

\documentclass{rapportECM}

% --- Liste des packages supplémentaires --- %
\usepackage{lipsum}
\usepackage{esint}

% ------------------------------------------ %

\begin{document}

% ------------------------------------------
%         Informations sur le rapport
% ------------------------------------------
\entreprise{Nom entreprise}
\logoentreprise{logos/entreprise.png} % Logo de l'entreprise

\titre{Titre de la mission} % Titre du fichier

\rendu{Rapport de stage de fin d'étude} % Nom du rendu

\eleve{Prénom Nom}
\PE{2020} % Promo entrante

\annee{3\textsuperscript{ème} année} % Année
\master{Master trop stylé} % Nom du master si besoin
\filiere{Filiere qui déchire tout} % Nom de la filière


\dates{date début - date fin}

% Informations tuteurs écoles
\tuteurecole{
  \textsc{Prénom Nom} \\
  prenom.nom@centrale-marseille.fr \\
} 

\tuteurentreprise{
  \textsc{Prénom Nom} \\
  prenom.nom@entreprise.fr \\
}

% ------------------------------------------
%               Initialisation
% ------------------------------------------
\fairemarges
\fairepagedegarde
\initfigures

%permet de laisser une page blanche après la page de garde (facultatif)
\pageblanche

% ------------------------------------------
%              Corps du document
% ------------------------------------------

% --- Table des matieres --- %
\tabledesmatieres

% ------------------------------------------
%              Remerciements
% ------------------------------------------
\ajoutsommaire{section}{Remerciements}
\pagestyle{remerciements} % permet d'écrire "Remerciements" dans l'en-tête

\vspace*{\stretch{1}}
\begin{center}
  \renewcommand{\abstractname}{Remerciements}
	\begin{abstract}
    \lipsum[1-2]
  \end{abstract}
\end{center}
\vspace*{\stretch{1}}
\newpage


% ------------------------------------------
%                 Résumé
% ------------------------------------------
\ajoutsommaire{section}{Résumé}
\pagestyle{resumeabstract} % permet d'écrire "Résumé & Abstract" dans l'en-tête

\vspace*{\stretch{1}}
\begin{center}
  \renewcommand{\abstractname}{Résumé}
	\begin{abstract}
    \lipsum[1-2]
  \end{abstract}
\end{center}
\vspace*{\stretch{1}}

% ------------------------------------------
%                 Abstract
% ------------------------------------------
\ajoutsommaire{section}{Abstract}
\vspace*{\stretch{1}}
\begin{center}
  \renewcommand{\abstractname}{Abstract}
	\begin{abstract}
    \lipsum[1-2]
  \end{abstract}
\end{center}
\vspace*{\stretch{1}}

\newpage

% ------------------------------------------
%              Introduction
% ------------------------------------------
\ajoutsommaire{section}{Introduction}
\pagestyle{introduction} % permet d'écrire "Introduction" dans l'en-tête

\section*{Introduction}
\lipsum[2-3]
\newpage
\pagestyle{document}

% ------------------------------------------
%                 Contenu
% ------------------------------------------

\section{Première section}

\lipsum[3-4]

\subsection{Sous partie}

\lipsum[3-4]

\section{Deuxième section}

\lipsum[2-5]

\section{Quelques commandes}

\insererfigure{images/centrale-photo.jpg}{0.6\textwidth}{Une photo de notre école}{mediatheque}
% 1- Chemin vers l'image
% 2- Hauteur de l'image (7cm ou bien 0.6\textwidth ==>60%)
% 3- Légénde sous l'image
% 4- Label associé à l'image pour pouvoir la citer



Ici je cite la figure \ref{fig: mediatheque} qui se trouve sur la page \pageref{fig: mediatheque}.\\

Ici je cite l'article \cite{Debauche2021}.

Ici j'écris une équation en mode ligne $\int_0^{+\infty} e^{-x^2} dx = \sqrt{\pi}$.

Et là mon équation va être mise en valeur au centre de la page :

$$
\oiint_{\mathcal{S}} \vec{A} \cdot \vec{dS} = \iiint_{\mathcal{V}} \textrm{div}(\vec{A}) \, d\tau
$$

\ajoutsommaire{section}{Conclusion}
\section*{Conclusion}
\lipsum[1-2]

% ------------------------------------------
%                 Glossaire
% ------------------------------------------
\newpage
\pagestyle{glossaire}
\ajoutsommaire{section}{Glossaire}
\section*{Glossaire}
\begin{itemize}
  \itemglos{Mot 1}{explication}
  \itemglos{Mot 2}{explication}
\end{itemize}
\pagebreak

% ------------------------------------------
%                Bibliographie
% ------------------------------------------
\newpage
\pagestyle{biblio}
\ajoutsommaire{section}{Bibliographie}
\section*{Bibliographie}
\printbibliography[ heading = none]
\pagebreak

% ------------------------------------------
%                   Annexes
% ------------------------------------------
\section*{Annexes}
\pagestyle{annexes}
\initannexes
\annexitem{mon annexe}
\annexitem{mon autre annexe}

\end{document}